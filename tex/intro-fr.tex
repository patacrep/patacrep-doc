\addentry{toc}{section}{\textbf{Introduction}}
\chapter*{Introduction}
\minitoc
\label{chap:introduction}
\chaptermark{Introduction}

Dans ce document, nous expliquons comment utiliser et tirer pleinement partie des outils et ressources proposés sur \url{http://www.patacrep.com} pour réaliser des recueils de chansons. Le projet \patacrep se décompose en plusieurs composant :
\begin{itemize}
   \item \patacrep est le moteur de compilation en ligne de commande permettant de créer les fichiers PDF ;
   \item \patagui est une interface graphique pour utiliser ce moteur ;
   \item \patadata est une base de donnée de l'ensemble des chansons réunies par la communautée ;
   \item \patawww est un site web offrant le meme type d'interface que \patagui. 
\end{itemize}

Voici une présentation rapide de ces divers composants.

\paragraph{\patacrep}
\patacrep est un projet libre et gratuit pour la production de recueils de tablatures. Il est rendu possible par le projet libre \songs, développé par \href{http://www.utdallas.edu/~hamlen}{Kevin W. Hamlem} et permettant d'écrire des recueils de tablatures avec \LaTeX\. \patacrep offre également la possibilité d'intégrer des partitions ou des extraits de partitions générés par \href{http://lilypond.org}{Lilypond}.

\patacrep est écrit en Python et en LaTeX, et fourni un utilitaire en ligne de commande que l'on peut appeler avec la commande \command{songbook}. L'interaction avec cet utilitaire passe par l'écriture d'un fichier de description d'un receuil (fichier .sb) ; et par un certain nombre d'options, détaillées dans la partie~\ref{}.

Vous pouvez par exemple consulter le fichier \href{http://www.patacrep.com/data/documents/songbook.pdf}{\file{songbook.pdf}} comprenant l'ensemble des chansons pour vous faire une idée du résultat avec le style par défaut.

\paragraph{\patadata}
\patadata est une base de données de tablatures, réalisée collaborativement par la communauté d'utilisateurs de \href{http://www.patacrep.com}{\patacrep}. Ces tablatures sont écrites dans des fichier \LaTeX\ portant l'extension .sg, et comportent parfois des partitions lilypond. L'organisation de \patadata et la syntaxe à utiliser pour y ajouter vos propres chansons sont décrits dans la partie~\ref{}.

\paragraph{\patagui}
Il s'agit d'une application graphique servant d'interface à l'utilitaire \command{songbook}. L'application permet de sélectionner en un clic les chansons à faire apparaître dans un recueil et fournit également un éditeur intégré pour modifier ou ajouter une nouvelle tablature.

\begin{nota}
Attention, par manque de temps et de developpeurs compétents, \patagui est pour le moment obsolète. Il est toujours possible de l'utiliser avec l'ancienne version de \patacrep, le \emph{Songbook 3.7.???}.

\todo{Ajouter la doc de l'ancienne version}
\end{nota}


\paragraph{\patawww}
\todo

\paragraph{Licences}
Tout le code est distribué sous licence \href{http://www.gnu.org/licenses/gpl.html}{GPLv2}. Tous les documents
et autres resources sont distribués sous une licence \href{http://creativecommons.org/}{Creative Commons CC-By-Sa}.
